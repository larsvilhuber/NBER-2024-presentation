
\begin{figure}[htb]
\centering
\caption{From Bertrand et al (2015), their Figure~I}\label{fig:bertrand}
\begin{tabular}{cc}
(a) & (b)\\
\includegraphics[width=0.4\textwidth]{../../outputs/jma_graph_rel_earn3_syn}&
\includegraphics[width=0.4\textwidth]{../../outputs/jma_graph_rel_earn3}\\
\multicolumn{2}{l}{\tiny \it Note: See text for details on computation and provenance.}
\end{tabular}
\end{figure}

An example for a successful validation request combined with a cautionary note for users of the 
synthetic data is illustrated by Figure~\ref{fig:bertrand}. In preparing \cite{Bertrand29012015}, the 
authors of that paper performed an analysis of the distribution of relative household income 
using a variety of datasets, including the \ac{SSB}. They obtained fairly robust results across a 
variety of datasets and time (see their Figure~III, reproduced here in 
Figure~\ref{fig:bertrand-qje-census}): there is a distinct break in the distribution of couples 
when the wife's income surpassed 50\%. However, the analysis with the \ac{SSB} produced a 
very different result, as illustrated in Figure~\ref{fig:bertrand}, Panel (a): there was no such 
break. The authors requested validation, using the protocols described above, and which the 
Census Bureau was able to accomplish in a very short time. 
The Census Bureau ran the same models on the confidential data, subjected the proposed 
publication statistics  to conventional statistical disclosure limitation (in this case just 
rounding and release in the form of a graph), and released Figure~\ref{fig:bertrand}, Panel (b), 
corresponding to Figure~I in  \cite{Bertrand29012015}. 
The results obtained when their 
analysis was replicated against the confidential files yielded a result that differed from the results obtained from the synthetic data, but which was  consistent with results obtained from the
other datasets.
The ``success'' alluded to earlier is on both sides of the interaction. The researchers were able 
to very quickly ascertain that their model, when tested against the confidential data, yielded a 
result in line with other results obtained from other data, and proceeded to publish their paper. 
The Census Bureau, in exchange, obtained valuable feedback on the 
quality of the synthesis models, which they were able to take into account for the next iteration 
of the data production cycle, and which is the statutory justification for the researchers' use of 
the validation process. The cautionary note is that while useful for exploring the data and 
for testing models, not every model will yield valid results on the synthetic data.%
\footnote{We thank Marianne Bertrand for allowing us to use this example, and for kindly 
having provided the graphs for from the analyses using the \ac{GSF} and the \ac{SSB}.}


\begin{figure}[hbt]
\centering
\caption{From Bertrand et al (2015)}\label{fig:bertrand-qje-census}
\includegraphics[width=0.5\textwidth]{../../outputs/Bertrand-QJE-2015-FigureIII}
\end{figure}



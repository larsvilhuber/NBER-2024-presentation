
\section{History of the Synthetic Data Server}
\input{../history.tex}


%\section{Validation}
%\input{validation.tex}

\section{Datasets}
\input{../datasets.tex}

\section{Usage and impact}
Since the start of the Synthetic Data Server project, account growth has been steady (see Figure~\ref{fig:accounts}). As of 2015 Q4, 186 accounts had been created on the server. In general, the rate of account requests increased subsequent to conference presentations or after specific events, such as the 2014 workshop (held by the University of Michigan NCRN node in collaboration with the U.S. Census Bureau and Cornell University) dedicated to teaching graduate students on how to use and leverage the \ac{SSB}.

\begin{figure}
\centering
\caption{Account creation and salient events\label{fig:accounts}}
\includegraphics{../../outputs/report_on_SDS_2021_ICES-accounts}
\end{figure}


\subsection{Feedback loop}
Both of the current data providers have incorporated feedback from users into their data products. The \ac{SSB} with which the \ac{SDS} was launched was itself the second public iteration, after the original release of v4.0. The growth of the supporting IT infrastructure, first from its predecessor to the original instantiation of the \ac{SDS}, and subsequently in the 2014 IT upgrade mentioned earlier, reflected the growing interest that followed adaptations. In the case of the \ac{SSB}, such feedback first lead to the incorporation of additional variables in v5.1 \citep{SSB5.1}, and subsequently to further enhancements in v6.0 \citep{SSB6}.
In the case of the \ac{SynLBD}, only one iteration has so far been released on the \ac{SDS}, but the key shortcomings in the structure of the SynLBD v2.0 \citep{SynLBD20} -- the absence of \ac{NAICS} codes, of any sort of geographic detail, and of indicators of firm structure -- have been reflected in the rejected access applications.
As a reminder, access is granted when the technical requests are feasibly satisfied by the data on the \ac{SDS}.
In the case of the SynLBD, we can quantify additional data requested that lead to applications being rejected.
Out of 100 applications for access to the \ac{SynLBD} received through 2015-08-10, 21 (21\%) were denied because they were not feasible using the synthetic data (this does not take into account applications that were partially feasible, which were generally approved). Of those denied,
\begin{itemize}
\item 6 (28.57\%) had requested firm-level variables,
\item 11 (52.38\%) had requested data to perform conditional geographic analysis, and
\item 1 (4.76\%) had requested data for specific \ac{NAICS} industries.
\end{itemize}
Such numbers, of course, do not take into account potential requestors that did not apply because a reading of the documentation revealed that the analysis was not feasible.

\subsection{Validation requests}

The data are by their nature preliminary, and users are discouraged from using results based
solely on the synthetic data. Validation requests are encouraged and free, subject to following
certain rules, outlined on the data providers' websites\footnote{\acf{SSB} Website at the U.S.
Census Bureau:
\url{http://www.census.gov/programs-surveys/sipp/methodology/sipp-synthetic-beta-data-product.html},
 \acf{SynLBD} at the U.S. Census Bureau:
\url{http://www.census.gov/ces/dataproducts/synlbd/}}. Generally, validation requires that
users provide all programs and auxiliary input files, documentation of the results similar to a
disclosure review request at \ac{FSRDC}, and that all programs run error-free
(replicability requirement). Results obtained from confidential data are subject to all the
disclosure avoidance protocols in effect at the time of their release. That being said, the
requirements are no more onerous than generic replication requirements, and turnaround may
be as short as one week.

An analysis of \ac{SynLBD} validation requests was performed as of 2015-08-10. Out of 79 projects, 5 had requested validation. A recurring issue has been that users are unfamiliar with the constraints imposed by the validation procedure, in particular that all such validation requests are treated as an authorized release of results from an analysis of confidential data, and are thus subject to review by the Census Bureau's \ac{DRB}. In particular, the \emph{quantity} of results requested surpasses not the ability of the confidential servers to compute, but rather is judged to be too high by the rules of the \ac{DRB} (60\% of validation requests).


\begin{figure}[htb]
\centering
\caption{From Bertrand et al (2015), their Figure~I}\label{fig:bertrand}
\begin{tabular}{cc}
(a) & (b)\\
\includegraphics[width=0.4\textwidth]{../../outputs/jma_graph_rel_earn3_syn}&
\includegraphics[width=0.4\textwidth]{../../outputs/jma_graph_rel_earn3}\\
\multicolumn{2}{l}{\tiny \it Note: See text for details on computation and provenance.}
\end{tabular}
\end{figure}

An example for a successful validation request combined with a cautionary note for users of the 
synthetic data is illustrated by Figure~\ref{fig:bertrand}. In preparing \cite{Bertrand29012015}, the 
authors of that paper performed an analysis of the distribution of relative household income 
using a variety of datasets, including the \ac{SSB}. They obtained fairly robust results across a 
variety of datasets and time (see their Figure~III, reproduced here in 
Figure~\ref{fig:bertrand-qje-census}): there is a distinct break in the distribution of couples 
when the wife's income surpassed 50\%. However, the analysis with the \ac{SSB} produced a 
very different result, as illustrated in Figure~\ref{fig:bertrand}, Panel (a): there was no such 
break. The authors requested validation, using the protocols described above, and which the 
Census Bureau was able to accomplish in a very short time. 
The Census Bureau ran the same models on the confidential data, subjected the proposed 
publication statistics  to conventional statistical disclosure limitation (in this case just 
rounding and release in the form of a graph), and released Figure~\ref{fig:bertrand}, Panel (b), 
corresponding to Figure~I in  \cite{Bertrand29012015}. 
The results obtained when their 
analysis was replicated against the confidential files yielded a result that differed from the results obtained from the synthetic data, but which was  consistent with results obtained from the
other datasets.
The ``success'' alluded to earlier is on both sides of the interaction. The researchers were able 
to very quickly ascertain that their model, when tested against the confidential data, yielded a 
result in line with other results obtained from other data, and proceeded to publish their paper. 
The Census Bureau, in exchange, obtained valuable feedback on the 
quality of the synthesis models, which they were able to take into account for the next iteration 
of the data production cycle, and which is the statutory justification for the researchers' use of 
the validation process. The cautionary note is that while useful for exploring the data and 
for testing models, not every model will yield valid results on the synthetic data.%
\footnote{We thank Marianne Bertrand for allowing us to use this example, and for kindly 
having provided the graphs for from the analyses using the \ac{GSF} and the \ac{SSB}.}


\begin{figure}[hbt]
\centering
\caption{From Bertrand et al (2015)}\label{fig:bertrand-qje-census}
\includegraphics[width=0.5\textwidth]{../../outputs/Bertrand-QJE-2015-FigureIII}
\end{figure}



%
%
%

More generally, the question as to the statistical precision of the results obtained from the synthetic data can be assessed. For each validation request, we captured model parameters from each of the models defined by researchers' programs, run against the confidential data ($\beta_{k,m}$) and the synthetic data ($\beta_{k,m}^*$). Because of the wide variety of models and estimation methods, this did not work always work well, nor does it distinguish between parameters of interest and nuisance or control parameters, for which precision may not be required or achieved. We then computed two measures of ``proximity''  as suggested by \cite{tas2006}: the proximity of  coefficients $\beta_{k,m}$ (estimated from the confidential data) and$\beta_{k,m}^*$ (from the synthetic data) , and a measure of the overlap of their confidence intervals.

To assess the proximity of parameter estimates, we compute
$$
t_{\Delta \beta_{k,m}} = \frac{\beta_{k,m} - \beta_{k,m}^*}{\sqrt{s_{k,m}^2 + s_{k,m}^{*2}}}
$$
and assess its statistical significance (90\% bilateral). The fraction of insignificant tests across all estimated models and parameters is an indicator of how close the synthetic and confidential models are under the estimated models.
%
To compute  the \emph{interval 
	overlap measure} $J_{k,m}$ for parameter $k$ in model $m$, consider the overlap of confidence intervals $(L,U)$ for $\beta_{k,m}$  and $(L^{*},U^{*})$ for $\beta_{k,m}^*$. Let $L^{over} = \max (L,L^{*} )$ and $U^{over} = \min (U,U^{*})$. Then the average overlap in confidence intervals is

$$
J_{k,m}^{*} = \frac{1}{2} \left [ \frac{U^{over} - L^{over}}{U-L} + \frac{U^{over} - L^{over}}{U^*-L ^*}        \right ]
$$
We then average $J_{k,m}^{*}$ over all estimated models and parameters, by validation request. 






\begin{Schunk}
\begin{Sinput}
> ssb.overlap <- read.csv(file.path(dataalt,"validation_overlap_ssb.csv"),stringsAsFactors = FALSE)
> ssb.results <- as.data.frame(summaryBy(cmp_overlap_0 ~ user,data=ssb.overlap, FUN = function(x) {c(Mean=mean(x),Median=median(x),quantile(x,probs=c(.75,.95)),Max=max(x),PctGrtThan0=100*mean(x>0))} ))
> names(ssb.results) <- sub("cmp_overlap_0.","",names(ssb.results))
> # the print.xtable puts out a tabular env, and removes the first column (user name)
> print.xtable(xtable(ssb.results[,-1]),file="table_coverage_ssb.tex",tabular.environment = "tabular",floating = FALSE)
\end{Sinput}
\end{Schunk}


\subsection{Prior and subsequent experience in the Census RDC}
\label{sec:prior}



On 2014-10-14, we researched what kind of exposure users of the \ac{SDS} had had to confidential data at the Census Bureau, by investigating whether they had prior or subsequent Census RDC projects. Firgure~\ref{fig:rdcUse} summarizes the results.


One of the (initially) unintended consequences was the use of the \ac{SDS} as a gateway to more in-depth use of confidential data, in particular data available through the \ac{FSRDC}. Conversely, once made aware of the availability of data similar in content to the data in the \ac{FSRDC}, researchers may wish to use the synthetic data, if it is sufficient for their research purposes. To assess the extent of such connections with the \ac{FSRDC}, we extracted a list of current and past users of the \ac{SDS} as of July 2014, and asked the Census Bureau to provide information on whether the users were known to the \ac{FSRDC} account management system. The response obtained in October 2014, is summarized in  Figure~\ref{fig:rdcUse}.%
\footnote{Private correspondence with Barbara Downs, Lead Research Data Center Administrator, October 14, 2014}
Of the 106 users we identified in this way, 14 were Census internal users, i.e., users who were actively engaged in ongoing Census projects, presumably related to the validation exercises themselves.
20 other users were also present in some form in the Census RDC system. 11 users had been authorized for at least one RDC project prior to their access to the \ac{SDS}.
More of interest, however, are the 3 users who obtained Census RDC access after their initial access to the \ac{SDS},
and the additional 6 who were still waiting for the approval of their RDC project, on average
516
days after having started their SDS project.
We don't have firm evidence of the relationship between the RDC projects and the SDS project, but from personal conversations of the authors with presenters at RDC conferences, at least some of the RDC projects were direct continuation of SDS projects, in domains that were not covered by the synthetic data, but the analysis for which was prepared for on the \ac{SDS}. The average delay between project start on \ac{SDS} and project start on RDC projects for those projects that were authorized was 400 days.


\begin{figure}
\centering
\caption{Connection between Census RDC usage and Synthetic Data Server}\label{fig:rdcUse}
\includegraphics{../../outputs/report_on_SDS_ISI2017-useRDCgraph}
\end{figure}

